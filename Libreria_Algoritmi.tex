% --------------------- LE GEMME DELLA LIBRERIA DEGLI ALGORITMI ----------------------

\chapter{Le gemme degli Algoritmi}

%TODO: Argomenti ancora da trattare in questo capitolo:

%TODO: C++20:
%TODO: ranges, concepts, constrained algorithms, coroutines, template parameter list, modules, ecc..

%TODO: execution policies
%TODO: optionals
%TODO: heap, set, queue, priority_queue
%TODO: std::accumulate, std::any_of, find, search
%TODO: copy_if, copy_n_code
%TODO: std_count, count_if
%TODO: scan, equal code, fill_if, fill_n_code
%TODO: std_generate, inner product, iota, permutations
%TODO: sort
%TODO: lexicographic compare
%TODO: minmax, max, min
%TODO: mismatch
%TODO: std_transform
%TODO: std_unique_code
%TODO: SFINAE e concepts

%TODO: std::clamp
%TODO: normal distribution, uniform real distribution? (non sono in <algorithm> quindi non ha senso metterli qui).

% ----------------------------- SECTION: INTRODUZIONE --------------------------------

\section{Introduzione}

\textsf{\small } \\

%TODO: L'importanza della libreria degli algoritmi.
%TODO: Mettere in un itemize la lista di tutte le operazioni: operazioni su sequenze non modificabili, oprazioni su sequeunze modificabili, ecc... (magari in parantesi come è scritto in inglese)
%TODO: dire che non le tratterò proprio tutte tutte, alcune sono molto simili.

%TODO: Inoltre, dire che tratterò argomenti anche che non fanno parte della libreria degli Algoritmi, ma che possono essere utili da utilizzare assieme agli algoritmi.

% ------------------ SECTION: OPERAZIONI SU SEQUENZE NON-MODIFICABILI  ---------------

% ------------------ SECTION: OPERAZIONI SU SEQUENZE MODIFICABILI --------------------

% ---------------------- SECTION: OPERAZIONI SU PARTIZIONI ---------------------------

% --------------------- SECTION: OPERAZIONI DI ORDINAMENTO ---------------------------

% ------------ SECTION: OPERAZIONI DI RICERCA BINARIA (BINARY SEARCH) ----------------

% --------------- SECTION: ALTRE OPERAZIONI DI ORDINAMENTO SUI RANGES ----------------

%TODO: questa parte nella sezione sul C++20

% ------------------------ SECTION: OPERAZIONI SUGLI INSIEMI -------------------------

% ------------------------- SECTION: OPERAZIONI SU HEAP ------------------------------

% ------------------------- SECTION: OPERAZIONI DI MIN/MAX ---------------------------

% ----------------------- SECTION: OPERAZIONI DI COMPARAZIONE ------------------------

% ---------------------- SECTION: OPERAZIONI SU PERMUTAZIONI -------------------------

% -------------------------- SECTION: OPERAZIONI NUMERICHE ---------------------------

% ------------------ SECTION: OPERAZIONI SU MEMORIA INIZIALIZZATA --------------------

% ------------------------------ SECTION: OPTIONALS ----------------------------------

%TODO: optionals in default parameters in functions.

% ------------------------- SECTION: EXECUTION POLICIES ------------------------------

% -------------------------------- SECTION: C++20 ------------------------------------

%TODO: La roba del C++20 come ultimo argomento del capitolo.
%TODO: ranges, concepts, constrained algorithms, coroutines, template parameter list, modules, ecc..
%TODO: non tutte queste c'entrano con la libreria degli algoritmi, le altre le potrei trattare nel capitolo 'Concetti Avanzati' (visto che è anche il più corto)
%TODO: tipo la libreria <coroutine>, <concepts>, ecc..

% ------------------------------ FINE CAPITOLO ---------------------------------------